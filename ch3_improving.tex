\documentclass[a4paper,twoside,10pt]{book}
% PAGE DIMENSIONS
% FOR A5 without scaling
%\usepackage[inner=18mm, outer=10mm,top=20mm,headsep=10mm,bottom=10mm,paperwidth=148mm,paperheight=210mm]{geometry}
% FOR 17x24cm
\usepackage[inner=30mm, outer=20mm,top=24mm,headsep=10mm,bottom=20mm,paperwidth=170mm,paperheight=240mm]{geometry}
% FOR A4
%\usepackage[inner=50mm, outer=40mm,top=60mm,headsep=10mm,bottom=57mm,paperwidth=210mm,paperheight=297mm]{geometry}
%\usepackage[inner=8.5mm, outer=12.5mm,top=10mm,headsep=5mm,bottom=10mm,paperwidth=141mm,paperheight=200mm]{geometry}
%\usepackage[inner=30mm, outer=23mm,top=29mm,headsep=10mm,bottom=23mm,paperwidth=210mm,paperheight=297mm]{geometry}
% FOR A5
%\usepackage[inner=25mm, outer=17mm,top=8mm,headsep=8mm,bottom=17mm,paperwidth=148mm,paperheight=210mm,includehead]{geometry}
% TABLES: full-width using X{10cm} column
\usepackage{tabularx}
% DRAWINGS (for chapter title)
\usepackage{tikz}
% TYPOGRAPHY: 
\usepackage[protrusion=true,final,factor=1500]{microtype}

% SOME extra math symbols
\usepackage{amssymb}

% TEXT on top of pictures
% ABS or REL positioning
% REL is better for scaled figures
\usepackage[rel]{overpic}
% CHEMICAL formulae like \ce{SiO2} and \ce{(NH4)2S2O8}
\usepackage[version=4]{mhchem}



% CAPTIONS of figures/table styling
\usepackage[font={stretch=1.1,small},labelfont=bf]{caption} % order matters: stretch, then size
%\usepackage[font={small,sf},labelfont=bf]{caption} % small sans-serif + bold

% FONT  SIZES
\usepackage{fix-cm} % FOR ANY font sizes, not predefined
\renewcommand\footnotesize{\fontsize{7pt}{7pt}\selectfont}
\renewcommand\small{\fontsize{6pt}{6pt}\selectfont}
\renewcommand\normalsize{\fontsize{9pt}{9pt}\selectfont}
\renewcommand\large{\fontsize{10pt}{10pt}\selectfont}
\renewcommand\Large{\fontsize{11pt}{11pt}\selectfont}
\renewcommand\Huge{\fontsize{24pt}{24pt}\selectfont}

%\renewcommand\small{\fontsize{10pt}{10pt}\selectfont}
%\renewcommand\normalsize{\fontsize{12pt}{12pt}\selectfont}
%\renewcommand\large{\fontsize{14pt}{14pt}\selectfont}
%\renewcommand\Large{\fontsize{16pt}{16pt}\selectfont}
%\renewcommand\Huge{\fontsize{36pt}{36pt}\selectfont}

% LINE spacing
\renewcommand{\baselinestretch}{1.25}

% FONTS for LaTeX/XeLaTeX
\usepackage{ifxetex}
\ifxetex
			\usepackage{mathspec}
			\usepackage{polyglossia}
			\setdefaultlanguage[variant=uk]{english}
			\defaultfontfeatures{Ligatures=TeX} % To support LaTeX quoting style
			\setmainfont{Minion Pro} 
			\setsansfont{Myriad Pro}
			\setmathsfont(Digits,Greek,Latin)[Numbers={Proportional}]{Minion Pro}
			\setmathrm{Minion Pro}
			\usepackage[italic]{mathastext}
			%\setmainfont[BoldFont={SwiftNeueLTW01-Bold},ItalicFont={SwiftNeueLTW01-Italic}]{SwiftNeueLTW01} 
			%\setsansfont{UniversLTW01-55Roman}
\else
			%\usepackage[T1]{fontenc}
			%\usepackage[utf8]{inputenc}
			\usepackage[english]{babel}
			%\usepackage{fouriernc} % UTOPIA + FOURIER
			%\usepackage{mathpple} % palatino
			%\usepackage[sc]{mathpazo}
			\usepackage[charter]{mathdesign} %utopia, garamond
			\usepackage[scaled]{helvet}
			%\usepackage{newtxtt}
			\def\ttdefault{cmvtt}
			%\usepackage[light]{roboto}
			%\renewcommand{\sfdefault}{ua1}
\fi

\usepackage{hyperref}
% HEADER/FOOTER
\usepackage{fancyhdr}
\renewcommand{\headrulewidth}{0pt}
\renewcommand{\footrulewidth}{0pt}
\fancyhf{}
\makeatletter
	\let\ps@plain\ps@empty
\makeatother
% PLAIN: for first page of chapter
\fancypagestyle{plain}{
	\renewcommand{\headrulewidth}{0pt}
	\fancyhf{}
	\fancyhead[RO]{
		\makebox[2cm][l]{
			\makebox[4cm][c]{\normalsize
				\hskip0.25em\phantom{\thepage}\phantom{XX}~~$\left|\vphantom{\int_a^b}\right.$~~\thepage\phantom{XX}
			}
		}
	}
	\fancyhead[LE]{
		\makebox[2cm][r]{
			\makebox[4cm][c]{\normalsize
				\hskip0.45em\phantom{XX}\thepage~~$\left|\vphantom{\int_a^b}\right.$~~\phantom{XX}\phantom{\thepage}
			}
		}
	}
}
% REGULAR pages
\fancyhead[RO]{
	\makebox[2cm][l]{
		\makebox[4cm][c]{\normalsize
			\hskip0.25em\phantom{\thepage}\nouppercase\rightmark~~$\left|\vphantom{\int_a^b}\right.$~~\thepage\phantom{\nouppercase\rightmark}
		}
	}
}
\fancyhead[LE]{
	\makebox[2cm][r]{
		\makebox[4cm][c]{\normalsize
			\hskip0.45em\phantom{\nouppercase\leftmark}\thepage~~$\left|\vphantom{\int_a^b}\right.$~~\nouppercase\leftmark\phantom{\thepage}
		}
	}
}

% MARGIN labels, except chapter=0 (intro; conclusion; bibliography etc.)
% AND \setcouter{chapter}{0} has to be set explicitely in required chapters
\fancyfoot[RO]{
	\ifnum\value{chapter}>0
	\begin{tikzpicture}[remember picture, overlay]
	\node[rounded corners=2mm,inner sep=3mm,anchor=north east,black,fill=black!15,draw=black!75] at ([xshift=2mm,yshift=-\arabic{chapter}*1.3cm-1.1cm]current page.north east) {\fontsize{1cm}{1cm}\selectfont\thechapter};
	\end{tikzpicture}
	\fi
}
\fancyfoot[LE]{
	\ifnum\value{chapter}>0
	\begin{tikzpicture}[remember picture, overlay]
	\node[rounded corners=2mm,inner sep=3mm,anchor=north west,black,fill=black!15,draw=black!75] at ([xshift=-2mm,yshift=-\arabic{chapter}*1.3cm-1.1cm]current page.north west) {\fontsize{1cm}{1cm}\selectfont\thechapter};
	\end{tikzpicture}
	\fi
}



\pagestyle{fancy}
% HEADER contents - chapter name and section name
%\renewcommand{\chaptermark}[1]{\markboth{\thechapter.\, #1}{}}
\renewcommand{\chaptermark}[1]{\markboth{#1}{}}
\renewcommand{\sectionmark}[1]{\markright{\thesection.\, #1}}


% FLOATING objects
\usepackage{float}
% PICTURES
\usepackage{graphicx}
% NOT USED
\usepackage{setspace}

% SPACINGS in lists
\usepackage{enumitem}
% VERICAL: topsep partopsep parsep itemsep
% HORIZONTAL: leftmargin rightmargin listparindent labelwidth labelsep itemindent
% GLOBAL: \setlist[enumerate]{labelsep=*, leftmargin=1.5pc}
\setlist{noitemsep}
%\setlist{nosep}




%% BIBLATEX
%% BIBliography: if authors <= MAXBIBNAMES - show all, ELSE: show only MINBIBNAMES et al.
%% TEXTCITE: if authors <= MAXCITENAMES - show all, ELSE: show only MINCITENAMES et al.
%% SORTCITES = sort numbers in ascending order
%% IGNORE/NOT IGNORE URL, DOI, EPRINT fields
%% BIBstyle = ieee (available)/ nature / science etc. (requires *.bbx file)
%% NICE: ieee+doi=true and phys+doi=false
%\usepackage[maxbibnames=3,minbibnames=1,backend=bibtex,maxcitenames=2,mincitenames=1,bibstyle=ieee,citestyle=numeric-comp,sorting=none,sortcites,url=false,doi=true,eprint=false]{biblatex}
%% REDUCE font size for bibliography (=useless waste of space)
%%\renewcommand*{\bibfont}{\footnotesize} %too small
%\renewcommand*{\bibfont}{\small}
%% SUPPRESS 'in' nefore journal name
%\renewbibmacro{in:}{}
%% LIST of bib files
%\addbibresource{library.bib}
%% STYLE of URL/DOI: Serif font instead of monospaced (mono looks ugly)
%\urlstyle{rm}
%
%% REdefine fullcite for personal publications list
%\DeclareCiteCommand{\fullcite}
%{\usebibmacro{prenote}}
%{\usedriver
%	{\defcounter{minnames}{6}%
%		\defcounter{maxnames}{9}}
%	{\thefield{entrytype}}.}
%{\multicitedelim}
%{\usebibmacro{postnote}}
%
%% FOOTnote on the chapter first page without footnotemark
%% RESULTS of this chapter were published as a paper
%\makeatletter
%    \def\blfootnote{\gdef\@thefnmark{}\@footnotetext}
%\makeatother


% TABLES: nicer rulers
% AND nicer spacing between lines in tables
\usepackage{booktabs}
\renewcommand{\arraystretch}{1.2}


% TITLES styling
\usepackage[toctitles,explicit,raggedright]{titlesec}
\newcommand*\chapterlabel{}
% CHAPTER in frontmatter|backmatter
\titleformat{name=\chapter,numberless}[display]
	{\normalfont\rmfamily\Huge\bfseries}{}{1ex}
	{\flushright{\chapterlabel#1}}
% CHAPTER in mainmatter
\titleformat{\chapter}
{\gdef\chapterlabel{}\normalfont\rmfamily\Huge\bfseries}
{\gdef\chapterlabel{}}{-10em}
{
	\flushright{
		\begin{tikzpicture}
			%\draw[help lines,step=5mm] (0,-3) grid (-\linewidth,3);
			\node[black!50,anchor=east,inner sep=0mm] (a) at (0,0) {\fontsize{7cm}{8cm}\selectfont\thechapter};
			\begin{scope}[cm={1,0,-0.6,0.15,(0,0)}].
					\node[transform shape,black!30,anchor=south,inner sep=0mm] at (a.south) {\fontsize{7cm}{8cm}\selectfont{}\thechapter};
					%\node[transform shape,black!30,anchor=south east,inner sep=0mm] at (a.south) {\fontsize{3cm}{8cm}\selectfont{}\chaptername};
			\end{scope}
			\node[black!50,anchor=east,inner sep=0mm] (a) at (0,0) {\fontsize{7cm}{8cm}\selectfont\thechapter};
			\node[black!35,anchor=east,inner sep=0.20mm,scale=0.98] at (0,0) {\fontsize{7cm}{8cm}\selectfont\thechapter};
			\node[black!20,anchor=east,inner sep=0.40mm,scale=0.96] at (0,0) {\fontsize{7cm}{8cm}\selectfont\thechapter};
			%\node[black!20,scale=0.97] at (a) {\fontsize{6cm}{8cm}\selectfont\thechapter};
			%\node[anchor=east,black!45] at (0,-1) {\resizebox{\linewidth}{!}{\chaptername}};
			%\node[anchor=north east,inner sep=0mm] at (a.north east) {\parbox{\linewidth}{\raggedleft\chapterlabel#1}};
			\node[anchor=east,inner sep=0mm] at (0,0) {\parbox{\linewidth}{\raggedleft\chapterlabel#1}};
		\end{tikzpicture}
		%\chapterlabel#1
	}
}
% SPACING: chapter by default uses \@makechapterhead with extra spacing before and after the chapter title
\titlespacing*{\chapter}{0pt}{-25pt}{30pt}

%\titleformat{\section}[block]{\Large}{\bfseries\thesection.\,\,#1}{1em}{}
%\titleformat{\subsection}[block]{\large}{\bfseries\thesubsection.\,\,#1}{1em}{}
%\titleformat{\subsection}{\no}{\itshape\thesubsection.\,#1}{1em}{}


% GREEK letters in section/chapter titles AND in PDF bookmarks
%\usepackage[artemisia]{textgreek}


% PART page picture
% DEFINE \partimage before the beginning of the part
\makeatletter
\newcommand{\partimage}[2][]{\gdef\@partimage{\includegraphics[#1]{#2}}}
% REDEFINED amsbook part code to include the \@partimage insertion command - copypaste from SeX
\def\@part[#1]#2{%
	\ifnum \c@secnumdepth >-2\relax \refstepcounter{part}%
	\addcontentsline{toc}{part}{\partname\ \thepart.
		\protect\enspace\protect\noindent#1}%
	\else
	\addcontentsline{toc}{part}{#1}\fi
	\begingroup\centering
	\ifnum \c@secnumdepth >-2\relax
	{\fontsize{\@xviipt}{22}\bfseries
		\partname\ \thepart} \vskip 20\p@ \fi
	\fontsize{\@xxpt}{25}\bfseries
	#1\vfil\@partimage\vfil\endgroup \newpage\thispagestyle{empty}}
\makeatother


\usepackage{listings}
\definecolor{mygreen}{rgb}{0,0.6,0}
\definecolor{mygray}{rgb}{0.5,0.5,0.5}
\definecolor{mymauve}{rgb}{0.58,0,0.82}
\usepackage{amsmath}
\usepackage{pgfplots}

\lstset{ 
	backgroundcolor=\color{white},   % choose the background color; you must add \usepackage{color} or \usepackage{xcolor}; should come as last argument
	basicstyle=\small\ttfamily,
	belowcaptionskip=0em,        % the size of the fonts that are used for the code
	belowskip=-2em,
	breakatwhitespace=false,         % sets if automatic breaks should only happen at whitespace
	breaklines=true,                 % sets automatic line breaking
	captionpos=b,                    % sets the caption-position to bottom
	commentstyle=\color{mygreen},    % comment style
	deletekeywords={...},            % if you want to delete keywords from the given language
	escapeinside={\%*}{*)},          % if you want to add LaTeX within your code
	extendedchars=true,              % lets you use non-ASCII characters; for 8-bits encodings only, does not work with UTF-8
	frame=single,	                   % adds a frame around the code
	keepspaces=true,                 % keeps spaces in text, useful for keeping indentation of code (possibly needs columns=flexible)
	keywordstyle=\color{blue},       % keyword style
	language=Python,                 % the language of the code
	morekeywords={*,...},            % if you want to add more keywords to the set
	numbers=none,                    % where to put the line-numbers; possible values are (none, left, right)
	numbersep=5pt,                   % how far the line-numbers are from the code
	numberstyle=\tiny\color{mygray}, % the style that is used for the line-numbers
	rulecolor=\color{black},         % if not set, the frame-color may be changed on line-breaks within not-black text (e.g. comments (green here))
	showspaces=false,                % show spaces everywhere adding particular underscores; it overrides 'showstringspaces'
	showstringspaces=false,          % underline spaces within strings only
	showtabs=false,                  % show tabs within strings adding particular underscores
	stepnumber=2,                    % the step between two line-numbers. If it's 1, each line will be numbered
	stringstyle=\color{mymauve},     % string literal style
	tabsize=2,	                   % sets default tabsize to 2 spaces
	title=\lstname 			          % show the filename of files included with \lstinputlisting; also try caption instead of title
}

\begin{document}
\chapter{Improving the way neural networks learn}
When a golf player is first learning to play golf, they usually spend most of their time developing a basic swing. Only gradually do they develop other shots, learning to chip, draw and fade the ball, building on and modifying their basic swing. In a similar way, up to now we've focused on understanding the backpropagation algorithm. It's our ``basic swing'', the foundation for learning in most work on neural networks. In this chapter I explain a suite of techniques which can be used to improve on our vanilla implementation of backpropagation, and so improve the way our networks learn.

The techniques we'll develop in this chapter include: a better choice of cost function, known as the cross-entropy cost function; four so-called ``regularization'' methods (L1 and L2 regularization, dropout, and artificial expansion of the training data), which make our networks better at generalizing beyond the training data; a better method for initializing the weights in the network; and a set of heuristics to help choose good hyper-parameters for the network. I'll also overview several other techniques in less depth. The discussions are largely independent of one another, and so you may jump ahead if you wish. We'll also implement many of the techniques in running code, and use them to improve the results obtained on the handwriting classification problem studied in Chapter 1.

Of course, we're only covering a few of the many, many techniques which have been developed for use in neural nets. The philosophy is that the best entree to the plethora of available techniques is in-depth study of a few of the most important. Mastering those important techniques is not just useful in its own right, but will also deepen your understanding of what problems can arise when you use neural networks. That will leave you well prepared to quickly pick up other techniques, as you need them.

\section{The cross-entropy cost function}
Most of us find it unpleasant to be wrong. Soon after beginning to learn the piano I gave my first performance before an audience. I was nervous, and began playing the piece an octave too low. I got confused, and couldn't continue until someone pointed out my error. I was very embarrassed. Yet while unpleasant, we also learn quickly when we're decisively wrong. You can bet that the next time I played before an audience I played in the correct octave! By contrast, we learn more slowly when our errors are less well-defined.

Ideally, we hope and expect that our neural networks will learn fast from their errors. Is this what happens in practice? To answer this question, let's look at a toy example. The example involves a neuron with just one input:

\begin{center}
	\includegraphics[width=0.5\linewidth]{figures/ch3/tikz28}
\end{center}
We'll train this neuron to do something ridiculously easy: take the input 1 to the output 0. Of course, this is such a trivial task that we could easily figure out an appropriate weight and bias by hand, without using a learning algorithm. However, it turns out to be illuminating to use gradient descent to attempt to learn a weight and bias. So let's take a look at how the neuron learns.

To make things definite, I'll pick the initial weight to be 0.6 and the initial bias to be 0.9. These are generic choices used as a place to begin learning, I wasn't picking them to be special in any way. The initial output from the neuron is 0.82, so quite a bit of learning will be needed before our neuron gets near the desired output, 0.0. 
%Click on ``Run'' in the bottom right corner below to see how the neuron learns an output much closer to 0.0. Note that this isn't a pre-recorded animation, your browser is actually computing the gradient, then using the gradient to update the weight and bias, and displaying the result.
The learning rate is $\eta$=0.15, which turns out to be slow enough that we can follow what's happening, but fast enough that we can get substantial learning in just a few seconds. The cost is the quadratic cost function, $C$, introduced back in Chapter 1. I'll remind you of the exact form of the cost function shortly, so there's no need to go and dig up the definition. %Note that you can run the animation multiple times by clicking on ``Run'' again.
\begin{center}%
	\includegraphics[width=0.75\linewidth]{./figures/ch3/animation_31}
\end{center}
As you can see, the neuron rapidly learns a weight and bias that drives down the cost, and gives an output from the neuron of about 0.09. That's not quite the desired output, 0.0, but it is pretty good. Suppose, however, that we instead choose both the starting weight and the starting bias to be 2.0. In this case the initial output is 0.98, which is very badly wrong. Let's look at how the neuron learns to output 0 in this case. %Click on ``Run'' again:
\begin{center}%
	\includegraphics[width=0.75\linewidth]{./figures/ch3/animation_32}
\end{center}
Although this example uses the same learning rate ($\eta=0.15$), we can see that learning starts out much more slowly. Indeed, for the first 150 or so learning epochs, the weights and biases don't change much at all. Then the learning kicks in and, much as in our first example, the neuron's output rapidly moves closer to 0.0.

This behavior is strange when contrasted to human learning. As I said at the beginning of this section, we often learn fastest when we're badly wrong about something. But we've just seen that our artificial neuron has a lot of difficulty learning when it's badly wrong -- far more difficulty than when it's just a little wrong. What's more, it turns out that this behavior occurs not just in this toy model, but in more general networks. Why is learning so slow? And can we find a way of avoiding this slowdown?

To understand the origin of the problem, consider that our neuron learns by changing the weight and bias at a rate determined by the partial derivatives of the cost function, $\partial{}C/\partial{}w$ and $\partial{}C/\partial{}b$. So saying ``learning is slow'' is really the same as saying that those partial derivatives are small. The challenge is to understand why they are small. To understand that, let's compute the partial derivatives. Recall that we're using the quadratic cost function, which, from Equation \ref{eq:6}, is given by
\begin{equation}
C = \frac{(y-a)^2}2,\label{eq:54}
\end{equation}
where $a$ is the neuron's output when the training input $x=1$ is used, and $y=0$ is the corresponding desired output. To write this more explicitly in terms of the weight and bias, recall that $a=\sigma{}(z)$, where $z=wx+b$. Using the chain rule to differentiate with respect to the weight and bias we get
\begin{eqnarray} 
\frac{\partial C}{\partial w} & = & (a-y)\sigma'(z) x = a \sigma'(z) \label{eq:55}\\
\frac{\partial C}{\partial b} & = & (a-y)\sigma'(z) = a \sigma'(z),
\label{eq:56}
\end{eqnarray}
where I have substituted $x=1$ and $y=0$. To understand the behaviour of these expressions, let's look more closely at the $\sigma{}'(z)$ term on the right-hand side. Recall the shape of the $\sigma{}$ function:

\begin{tikzpicture}[scale=0.75]
\begin{axis}[title={Sigmoid function}]
\addplot[blue,domain=-5:5,samples=81]{1/(1+exp(-x)};
\end{axis}
\end{tikzpicture}

We can see from this graph that when the neuron's output is close to 1, the curve gets very flat, and so $\sigma{}'(z)$ gets very small. Equations \ref{eq:55} and \ref{eq:56} then tell us that $\partial{}C/\partial{}w$ and $\partial{}C/\partial{}b$ get very small. This is the origin of the learning slowdown. What's more, as we shall see a little later, the learning slowdown occurs for essentially the same reason in more general neural networks, not just the toy example we've been playing with.

\section{Introducing the cross-entropy cost function}
How can we address the learning slowdown? It turns out that we can solve the problem by replacing the quadratic cost with a different cost function, known as the cross-entropy. To understand the cross-entropy, let's move a little away from our super-simple toy model. We'll suppose instead that we're trying to train a neuron with several input variables, $x_1,x_2,\ldots$, corresponding weights $w_1,w_2,\ldots$, and a bias, $b$:
\begin{center}
	\includegraphics[scale=0.5]{./figures/ch3/tikz29}
\end{center}
The output from the neuron is, of course, $a=\sigma(z)$, where $z=\sum_jw_jb_j+b$ is the weighted sum of the inputs. We define the cross-entropy cost function for this neuron by
%C=-1n∑x[ylna+(1-y)ln(1-a)],(57)
\begin{equation}
	C = -\frac{1}{n} \sum_x \left[y \ln a + (1-y ) \ln (1-a) \right],
	\label{eq:57}
\end{equation}
where $n$ is the total number of items of training data, the sum is over all training inputs, $x$, and $y$ is the corresponding desired output.

It's not obvious that the expression (\ref{eq:57}) fixes the learning slowdown problem. In fact, frankly, it's not even obvious that it makes sense to call this a cost function! Before addressing the learning slowdown, let's see in what sense the cross-entropy can be interpreted as a cost function.

Two properties in particular make it reasonable to interpret the cross-entropy as a cost function. First, it's non-negative, that is, $C>0$. To see this, notice that: (a) all the individual terms in the sum in (\ref{eq:57}) are negative, since both logarithms are of numbers in the range 0 to 1; and (b) there is a minus sign out the front of the sum.

Second, if the neuron's actual output is close to the desired output for all training inputs, x, then the cross-entropy will be close to zero\footnote{To prove this I will need to assume that the desired outputs y are all either 0 or 1. This is usually the case when solving classification problems, for example, or when computing Boolean functions. To understand what happens when we don't make this assumption, see the exercises at the end of this section.}. To see this, suppose for example that y=0 and a≈0 for some input $x$. This is a case when the neuron is doing a good job on that input. We see that the first term in the expression (57) for the cost vanishes, since $y=0$, while the second term is just $-\ln(1-a)\approx0$. A similar analysis holds when $y=1$ and $a\approx1$. And so the contribution to the cost will be low provided the actual output is close to the desired output.

Summing up, the cross-entropy is positive, and tends toward zero as the neuron gets better at computing the desired output, $y$, for all training inputs, $x$. These are both properties we'd intuitively expect for a cost function. Indeed, both properties are also satisfied by the quadratic cost. So that's good news for the cross-entropy. But the cross-entropy cost function has the benefit that, unlike the quadratic cost, it avoids the problem of learning slowing down. To see this, let's compute the partial derivative of the cross-entropy cost with respect to the weights. We substitute $a=\sigma(z)$ into (\ref{eq:57}), and apply the chain rule twice, obtaining:
%\partial{}C\partial{}wj==-1n∑x(y\sigma(z)-(1-y)1-\sigma(z))\partial{}\sigma\partial{}wj-1n∑x(y\sigma(z)-(1-y)1-\sigma(z))\sigma'(z)x_j.(58)(59)
\begin{equation}
	\frac{\partial C}{\partial w_j}  =  -\frac{1}{n} \sum_x \left(
	\frac{y }{\sigma(z)} -\frac{(1-y)}{1-\sigma(z)} \right)
	\frac{\partial \sigma}{\partial w_j} = 
 -\frac{1}{n} \sum_x \left( 
	\frac{y}{\sigma(z)} 
	-\frac{(1-y)}{1-\sigma(z)} \right)\sigma'(z) x_j.
	\label{eq:59}
\end{equation}Putting everything over a common denominator and simplifying this becomes:
\begin{equation}
	\frac{\partial C}{\partial w_j}  =  \frac{1}{n}
	\sum_x \frac{\sigma'(z) x_j}{\sigma(z) (1-\sigma(z))}
	(\sigma(z)-y).
	\label{eq:60}
\end{equation}
%\partial{}C\partial{}wj=1n∑x\sigma'(z)x_j\sigma(z)(1-\sigma(z))(\sigma(z)-y).(60)
Using the definition of the sigmoid function, $\sigma(z)=1/(1+e^{-z})$, and a little algebra we can show that $\sigma'(z)=\sigma(z)(1-\sigma(z))$. I'll ask you to verify this in an exercise below, but for now let's accept it as given. We see that the $\sigma'(z)$ and $\sigma(z)(1-\sigma(z))$ terms cancel in the equation just above, and it simplifies to become:
%\partial{}C\partial{}wj=1n∑xx_j(\sigma(z)-y).(61)
\begin{equation}
	\frac{\partial C}{\partial w_j} =  \frac{1}{n} \sum_x x_j(\sigma(z)-y).
	\label{eq:61}
\end{equation}
This is a beautiful expression. It tells us that the rate at which the weight learns is controlled by $\sigma(z)-y$, i.e., by the error in the output. The larger the error, the faster the neuron will learn. This is just what we'd intuitively expect. In particular, it avoids the learning slowdown caused by the $\sigma'(z)$ term in the analogous equation for the quadratic cost, Equation(\ref{eq:55}). When we use the cross-entropy, the $\sigma'(z)$ term gets canceled out, and we no longer need worry about it being small. This cancellation is the special miracle ensured by the cross-entropy cost function. Actually, it's not really a miracle. As we'll see later, the cross-entropy was specially chosen to have just this property.

In a similar way, we can compute the partial derivative for the bias. I won't go through all the details again, but you can easily verify that
\begin{equation}
	\frac{\partial C}{\partial b} = \frac{1}{n} \sum_x (\sigma(z)-y).
	\label{eq:62}
\end{equation}
%\partial{}C\partial{}b=1n∑x(\sigma(z)-y).(62)
Again, this avoids the learning slowdown caused by the $\sigma'(z)$ term in the analogous equation for the quadratic cost, Equation (\ref{eq:56}).
\subsection*{Exercise}
\begin{itemize}
	\item Verify that $\sigma'(z) = \sigma(z)(1-\sigma(z))$
\end{itemize}
Let's return to the toy example we played with earlier, and explore what happens when we use the cross-entropy instead of the quadratic cost. To re-orient ourselves, we'll begin with the case where the quadratic cost did just fine, with starting weight 0.6 and starting bias 0.9
%. Press "Run" to see what happens when we replace the quadratic cost by the cross-entropy
:
\begin{center}%
	\includegraphics[width=0.75\linewidth]{./figures/ch3/animation_33}
\end{center}
Unsurprisingly, the neuron learns perfectly well in this instance, just as it did earlier. And now let's look at the case where our neuron got stuck before ([link], for comparison), with the weight and bias both starting at 2.0:
\begin{center}%
	\includegraphics[width=0.75\linewidth]{./figures/ch3/animation_34}
\end{center}
Success! This time the neuron learned quickly, just as we hoped. If you observe closely you can see that the slope of the cost curve was much steeper initially than the initial flat region on the corresponding curve for the quadratic cost. It's that steepness which the cross-entropy buys us, preventing us from getting stuck just when we'd expect our neuron to learn fastest, i.e., when the neuron starts out badly wrong.

I didn't say what learning rate was used in the examples just illustrated. Earlier, with the quadratic cost, we used $\eta=0.15$. Should we have used the same learning rate in the new examples? In fact, with the change in cost function it's not possible to say precisely what it means to use the ``same'' learning rate; it's an apples and oranges comparison. For both cost functions I simply experimented to find a learning rate that made it possible to see what is going on. If you're still curious, despite my disavowal, here's the lowdown: I used $\eta=0.005$ in the examples just given.

You might object that the change in learning rate makes the graphs above meaningless. Who cares how fast the neuron learns, when our choice of learning rate was arbitrary to begin with?! That objection misses the point. The point of the graphs isn't about the absolute speed of learning. It's about how the speed of learning changes. In particular, when we use the quadratic cost learning is slower when the neuron is unambiguously wrong than it is later on, as the neuron gets closer to the correct output; while with the cross-entropy learning is faster when the neuron is unambiguously wrong. Those statements don't depend on how the learning rate is set.

We've been studying the cross-entropy for a single neuron. However, it's easy to generalize the cross-entropy to many-neuron multi-layer networks. In particular, suppose $y=y_1,y_2,\ldots$ are the desired values at the output neurons, i.e., the neurons in the final layer, while $a^L_1,a^L_2,\ldots$ are the actual output values. Then we define the cross-entropy by
%C=-1n∑x∑j[yjlna^L_j+(1-yj)ln(1-a^L_j)].(63)
\begin{equation}
	\sum_j \left[y_j \ln a^L_j + (1-y_j) \ln (1-a^L_j) \right].
	\label{eq:63}
\end{equation}
This is the same as our earlier expression, Equation (\ref{eq:57}), except now we've got the $\sum_j$ summing over all the output neurons. I won't explicitly work through a derivation, but it should be plausible that using the expression (\ref{eq:63}) avoids a learning slowdown in many-neuron networks. If you're interested, you can work through the derivation in the problem below.

Incidentally, I'm using the term ``cross-entropy'' in a way that has confused some early readers, since it superficially appears to conflict with other sources. In particular, it's common to define the cross-entropy for two probability distributions, $p_j$ and $q_j$, as $\sum_jp_j\ln{}q_j$. This definition may be connected to (\ref{eq:57}), if we treat a single sigmoid neuron as outputting a probability distribution consisting of the neuron's activation $a$ and its complement $1-a$.

However, when we have many sigmoid neurons in the final layer, the vector $a^L_j$ of activations don't usually form a probability distribution. As a result, a definition like $\sum_j p_j \ln q_j$ doesn't even make sense, since we're not working with probability distributions. Instead, you can think of (\ref{eq:63}) as a summed set of per-neuron cross-entropies, with the activation of each neuron being interpreted as part of a two-element probability distribution\footnote{Of course, in our networks there are no probabilistic elements, so they're not really probabilities.}. In this sense, (\ref{eq:63}) is a generalization of the cross-entropy for probability distributions.

When should we use the cross-entropy instead of the quadratic cost? In fact, the cross-entropy is nearly always the better choice, provided the output neurons are sigmoid neurons. To see why, consider that when we're setting up the network we usually initialize the weights and biases using some sort of randomization. It may happen that those initial choices result in the network being decisively wrong for some training input - that is, an output neuron will have saturated near 1, when it should be 0, or vice versa. If we're using the quadratic cost that will slow down learning. It won't stop learning completely, since the weights will continue learning from other training inputs, but it's obviously undesirable.

\subsection*{Exercises}
\begin{itemize}
	\item One gotcha with the cross-entropy is that it can be difficult at first to remember the respective roles of the ys and the as. It's easy to get confused about whether the right form is $$-[y \ln a + (1-y) \ln (1-a)].$$ What happens to the second of these expressions when $y=0$ or 1? Does this problem afflict the first expression? Why or why not?
	\item In the single-neuron discussion at the start of this section, I argued that the cross-entropy is small if $\sigma(z)\approx y$ for all training inputs. The argument relied on $y$ being equal to either 0 or 1. This is usually true in classification problems, but for other problems (e.g., regression problems) $y$ can sometimes take values intermediate between 0 and 1. Show that the cross-entropy is still minimized when $\sigma(z)=y$ for all training inputs. When this is the case the cross-entropy has the value:
	\begin{equation}
		C = -\frac{1}{n} \sum_x [y \ln y+(1-y) \ln(1-y)].
		\label{eq:64}
	\end{equation}
	The quantity $-[y \ln y+(1-y) \ln(1-y)]$ is sometimes known as the \textit{binary entropy}.
\end{itemize}
\subsection*{Problems}
\begin{itemize}
	\item \textbf{Many-layer multi-neuron networks} In the notation introduced in the last chapter, show that for the quadratic cost the partial derivative with respect to weights in the output layer is
	\begin{equation}
		\frac{\partial C}{\partial w^L_{jk}}  = \frac{1}{n}
		\sum_x a^{L-1}_k  (a^L_j-y_j) \sigma'(z^L_j).
		\label{eq:65}
	\end{equation}%\partial{}C\partial{}w^L_{jk}=1n∑xaL-1k(a^L_j-yj)\sigma'(z^L_j).(65)
	The term $\sigma'(z^L_j)$ causes a learning slowdown whenever an output neuron saturates on the wrong value. Show that for the cross-entropy cost the output error 
	$\delta^L$ for a single training example $x$ is given by
	\begin{equation}
		\delta^L = a^L - y.\label{eq:66}
	\end{equation}%δL=aL-y.(66)
	Use this expression to show that the partial derivative with respect to the weights in the output layer is given by
	\begin{equation}
		\frac{\partial C}{\partial w^L_{jk}}  =  \frac{1}{n} \sum_x 	a^{L-1}_k  (a^L_j-y_j).
		\label{eq:67}
	\end{equation}%\partial{}C\partial{}w^L_{jk}=1n∑xaL-1k(a^L_j-yj).(67)
	The $\sigma'(z^L_j)$ term has vanished, and so the cross-entropy avoids the problem of learning slowdown, not just when used with a single neuron, as we saw earlier, but also in many-layer multi-neuron networks. A simple variation on this analysis holds also for the biases. If this is not obvious to you, then you should work through that analysis as well.
	\item \textbf{Using the quadratic cost when we have linear neurons in the output layer} Suppose that we have a many-layer multi-neuron network. Suppose all the neurons in the final layer are linear neurons, meaning that the sigmoid activation function is not applied, and the outputs are simply $a^L_j=z^L_j$. Show that if we use the quadratic cost function then the output error $\delta^L$ for a single training example $x$ is given by
	\begin{equation}
		\delta^L = a^L-y.
		\label{eq:68}
	\end{equation}%δL=aL-y.(68)
	Similarly to the previous problem, use this expression to show that the partial derivatives with respect to the weights and biases in the output layer are given by
	\begin{eqnarray}
	\frac{\partial C}{\partial w^L_{jk}} & = & \frac{1}{n} \sum_x a^{L-1}_k  (a^L_j-y_j)\\
	\frac{\partial C}{\partial b^L_{j}} & = & \frac{1}{n} \sum_x (a^L_j-y_j).
	\label{eq:70}
	\end{eqnarray}
	%\partial{}C\partial{}w^L_{jk}\partial{}C\partial{}b^L_j==1n∑xaL-1k(a^L_j-yj)1n∑x(a^L_j-yj).(69)(70)
	This shows that if the output neurons are linear neurons then the quadratic cost will not give rise to any problems with a learning slowdown. In this case the quadratic cost is, in fact, an appropriate cost function to use.
\end{itemize}
\section{Using the cross-entropy to classify MNIST digits}
The cross-entropy is easy to implement as part of a program which learns using gradient descent and backpropagation. We'll do that later in the chapter, developing an improved version of our earlier program for classifying the MNIST handwritten digits, \texttt{network.py}. The new program is called \texttt{network2.py}, and incorporates not just the cross-entropy, but also several other techniques developed in this chapter\footnote{The code is available on [link]GitHub.}. For now, let's look at how well our new program classifies MNIST digits. As was the case in Chapter 1, we'll use a network with 30 hidden neurons, and we'll use a mini-batch size of 10. We set the learning rate to $\eta=0.5$\footnote{In Chapter 1 we used the quadratic cost and a learning rate of $\eta=3.0$. As discussed above, it's not possible to say precisely what it means to use the ``same'' learning rate when the cost function is changed. For both cost functions I experimented to find a learning rate that provides near-optimal performance, given the other hyper-parameter choices. \newline There is, incidentally, a very rough general heuristic for relating the learning rate for the cross-entropy and the quadratic cost. As we saw earlier, the gradient terms for the quadratic cost have an extra $\sigma' = \sigma(1-\sigma)$ term in them. Suppose we average this over values for $\sigma$, $\int_0^1\mathrm{d}\sigma \sigma(1-\sigma)=1/6$. We see that (very roughly) the quadratic cost learns an average of 6 times slower, for the same learning rate. This suggests that a reasonable starting point is to divide the learning rate for the quadratic cost by 6. Of course, this argument is far from rigorous, and shouldn't be taken too seriously. Still, it can sometimes be a useful starting point.} and we train for 30 epochs. The interface to \texttt{network2.py} is slightly different than \texttt{network.py}, but it should still be clear what is going on. You can, by the way, get documentation about \texttt{network2.py}'s interface by using commands such as \texttt{help(network2.Network.SGD)} in a Python shell.
\begin{lstlisting}
>>> import mnist_loader
>>> training_data, validation_data, test_data = mnist_loader.load_data_wrapper()
>>> import network2
>>> net = network2.Network([784, 30, 10], cost=network2.CrossEntropyCost)
>>> net.large_weight_initializer()
>>> net.SGD(training_data, 30, 10, 0.5, evaluation_data=test_data, monitor_evaluation_accuracy=True)
\end{lstlisting}	
Note, by the way, that the \texttt{net.large\_weight\_initializer()} command is used to initialize the weights and biases in the same way as described in Chapter 1. We need to run this command because later in this chapter we'll change the default weight initialization in our networks. The result from running the above sequence of commands is a network with 95.49 percent accuracy. This is pretty close to the result we obtained in Chapter 1, 95.42 percent, using the quadratic cost.
	
Let's look also at the case where we use 100 hidden neurons, the cross-entropy, and otherwise keep the parameters the same. In this case we obtain an accuracy of 96.82 percent. That's a substantial improvement over the results from Chapter 1, where we obtained a classification accuracy of 96.59 percent, using the quadratic cost. That may look like a small change, but consider that the error rate has dropped from 3.41 percent to 3.18 percent. That is, we've eliminated about one in fourteen of the original errors. That's quite a handy improvement.
	
It's encouraging that the cross-entropy cost gives us similar or better results than the quadratic cost. However, these results don't conclusively prove that the cross-entropy is a better choice. The reason is that I've put only a little effort into choosing hyper-parameters such as learning rate, mini-batch size, and so on. For the improvement to be really convincing we'd need to do a thorough job optimizing such hyper-parameters. Still, the results are encouraging, and reinforce our earlier theoretical argument that the cross-entropy is a better choice than the quadratic cost.
	
This, by the way, is part of a general pattern that we'll see through this chapter and, indeed, through much of the rest of the book. We'll develop a new technique, we'll try it out, and we'll get ``improved'' results. It is, of course, nice that we see such improvements. But the interpretation of such improvements is always problematic. They're only truly convincing if we see an improvement after putting tremendous effort into optimizing all the other hyper-parameters. That's a great deal of work, requiring lots of computing power, and we're not usually going to do such an exhaustive investigation. Instead, we'll proceed on the basis of informal tests like those done above. Still, you should keep in mind that such tests fall short of definitive proof, and remain alert to signs that the arguments are breaking down.
	
By now, we've discussed the cross-entropy at great length. Why go to so much effort when it gives only a small improvement to our MNIST results? Later in the chapter we'll see other techniques -- notably, regularization -- which give much bigger improvements. So why so much focus on cross-entropy? Part of the reason is that the cross-entropy is a widely-used cost function, and so is worth understanding well. But the more important reason is that neuron saturation is an important problem in neural nets, a problem we'll return to repeatedly throughout the book. And so I've discussed the cross-entropy at length because it's a good laboratory to begin understanding neuron saturation and how it may be addressed.

\section{What does the cross-entropy mean? Where does it come from?}
Our discussion of the cross-entropy has focused on algebraic analysis and practical implementation. That's useful, but it leaves unanswered broader conceptual questions, like: what does the cross-entropy mean? Is there some intuitive way of thinking about the cross-entropy? And how could we have dreamed up the cross-entropy in the first place?

Let's begin with the last of these questions: what could have motivated us to think up the cross-entropy in the first place? Suppose we'd discovered the learning slowdown described earlier, and understood that the origin was the $\sigma'(z)$ terms in Equations (\ref{eq:55}) and (\ref{eq:56}). After staring at those equations for a bit, we might wonder if it's possible to choose a cost function so that the $\sigma'(z)$ term disappeared. In that case, the cost $C=C_x$ for a single training example $x$ would satisfy
\begin{eqnarray} 
\frac{\partial C}{\partial w_j} & = & x_j(a-y) \label{eq:71}\\
\frac{\partial C}{\partial b } & = & (a-y).
\label{eq:72}\end{eqnarray}
%\partial{}C\partial{}wj\partial{}C\partial{}b==x_j(a−y)(a−y).(71)(72)
If we could choose the cost function to make these equations true, then they would capture in a simple way the intuition that the greater the initial error, the faster the neuron learns. They'd also eliminate the problem of a learning slowdown. In fact, starting from these equations we'll now show that it's possible to derive the form of the cross-entropy, simply by following our mathematical noses. To see this, note that from the chain rule we have
\begin{equation}
	\frac{\partial C}{\partial b} = \frac{\partial C}{\partial a} \sigma'(z).
	\label{eq:73}
\end{equation}
%\partial{}C\partial{}b=\partial{}C\partial{}a\sigma'(z).(73)
Using $\sigma'(z)=\sigma(z)(1−\sigma(z))=a(1−a)$ the last equation becomes
\begin{eqnarray}
\frac{\partial C}{\partial b} = \frac{\partial C}{\partial a} 
a(1-a).
\label{eq:74}\end{eqnarray}
%\partial{}C\partial{}b=\partial{}C\partial{}aa(1−a).(74)
Comparing to Equation~\ref{eq:72} we obtain
\begin{eqnarray}
\frac{\partial C}{\partial a} = \frac{a-y}{a(1-a)}. \label{eq:75}\end{eqnarray}
Integrating this expression with respect to $a$ gives
\begin{eqnarray}
C = -[y \ln a + (1-y) \ln (1-a)]+ {\rm constant},
\label{eq:76}\end{eqnarray}
%C=−[ylna+(1−y)ln(1−a)]+constant,(76)
for some constant of integration. This is the contribution to the cost from a single training example, $x$. To get the full cost function we must average over training examples, obtaining
\begin{eqnarray}
C = -\frac{1}{n} \sum_x [y \ln a +(1-y) \ln(1-a)] + {\rm constant},
\label{eq:77}\end{eqnarray}
where the constant here is the average of the individual constants for each training example. And so we see that Equations (\ref{eq:71}) and (\ref{eq:72}) uniquely determine the form of the cross-entropy, up to an overall constant term. The cross-entropy isn't something that was miraculously pulled out of thin air. Rather, it's something that we could have discovered in a simple and natural way.

What about the intuitive meaning of the cross-entropy? How should we think about it? Explaining this in depth would take us further afield than I want to go. However, it is worth mentioning that there is a standard way of interpreting the cross-entropy that comes from the field of information theory. Roughly speaking, the idea is that the cross-entropy is a measure of surprise. In particular, our neuron is trying to compute the function $x\to y=y(x)$. But instead it computes the function $x \to a=a(x)$. Suppose we think of a as our neuron's estimated probability that $y$ is 1, and $1-a$ is the estimated probability that the right value for $y$ is 0. Then the cross-entropy measures how ``surprised'' we are, on average, when we learn the true value for $y$. We get low surprise if the output is what we expect, and high surprise if the output is unexpected. Of course, I haven't said exactly what ``surprise'' means, and so this perhaps seems like empty verbiage. But in fact there is a precise information-theoretic way of saying what is meant by surprise. Unfortunately, I don't know of a good, short, self-contained discussion of this subject that's available online. But if you want to dig deeper, then Wikipedia contains a [link]brief summary that will get you started down the right track. And the details can be filled in by working through the materials about the Kraft inequality in chapter 5 of the book about information theory by [link]Cover and Thomas.

\subsection*{Problem}
\begin{itemize}
	\item We've discussed at length the learning slowdown that can occur when output neurons saturate, in networks using the quadratic cost to train. Another factor that may inhibit learning is the presence of the $x_j$ term in Equation (\ref{eq:61}). Because of this term, when an input $x_j$ is near to zero, the corresponding weight $w_j$ will learn slowly. Explain why it is not possible to eliminate the $x_j$ term through a clever choice of cost function.
\end{itemize}

\section{Softmax}
In this chapter we'll mostly use the cross-entropy cost to address the problem of learning slowdown. However, I want to briefly describe another approach to the problem, based on what are called \textit{softmax} layers of neurons. We're not actually going to use softmax layers in the remainder of the chapter, so if you're in a great hurry, you can skip to the next section. However, softmax is still worth understanding, in part because it's intrinsically interesting, and in part because we'll use softmax layers in Chapter 6, in our discussion of deep neural networks.

The idea of softmax is to define a new type of output layer for our neural networks. It begins in the same way as with a sigmoid layer, by forming the weighted inputs\footnote{In describing the softmax we'll make frequent use of notation introduced in the last chapter. You may wish to revisit that chapter if you need to refresh your memory about the meaning of the notation.} $z^L_j=\sum_kw^L_{jk}a^{L−1}_k+b^L_j$. However, we don't apply the sigmoid function to get the output. Instead, in a softmax layer we apply the so-called softmax function to the $z^L_j$. According to this function, the activation $a^L_j$ of the $j$-th output neuron is
\begin{equation}
	a^L_j = \frac{e^{x^L_j}}{\sum_ke^{z^L_k}}
	\label{eq:78}
\end{equation}
%a^L_j=ez^L_j∑kezLk,(78)
where in the denominator we sum over all the output neurons.

If you're not familiar with the softmax function, Equation (\ref{eq:78}) may look pretty opaque. It's certainly not obvious why we'd want to use this function. And it's also not obvious that this will help us address the learning slowdown problem. To better understand Equation (\ref{eq:78}), suppose we have a network with four output neurons, and four corresponding weighted inputs, which we'll denote $z^L_1,z^L_2,z^L_3,$ and $z^L_4$. Figure\ref{fig:softmax} shows  a graph of the corresponding output activations for different inputs\footnote{This paragraph is an adaptation of an animation from online version of the book.}.
\begin{figure}

\includegraphics[width=\linewidth]{./figures/ch3/animation_softmax}
\caption{Equation~\ref{eq:78} for different fixed values of $z^L_{1,2,3}$ and variable $z^L_4$. $L$ index avoided for clarity. }
\label{fig:softmax}
\end{figure}

As you increase $z^L_4$, you'll see an increase in the corresponding output activation, $a^L_4$, and a decrease in the other output activations. Similarly, if you decrease $z^L_4$ then $a^L_4$ will decrease, and all the other output activations will increase. In fact, if you look closely, you'll see that in both cases the total change in the other activations exactly compensates for the change in $a^L_4$. The reason is that the output activations are guaranteed to always sum up to 1, as we can prove using Equation (\ref{eq:78}) and a little algebra:
\begin{eqnarray}
\sum_j a^L_j & = & \frac{\sum_j e^{z^L_j}}{\sum_k e^{z^L_k}} = 1.
\label{eq:79}\end{eqnarray}
%∑ja^L_j=∑jez^L_j∑kezLk=1.(79)
As a result, if $a^L_4$ increases, then the other output activations must decrease by the same total amount, to ensure the sum over all activations remains 1. And, of course, similar statements hold for all the other activations.

Equation (\ref{eq:78}) also implies that the output activations are all positive, since the exponential function is positive. Combining this with the observation in the last paragraph, we see that the output from the softmax layer is a set of positive numbers which sum up to 1. In other words, the output from the softmax layer can be thought of as a probability distribution.

The fact that a softmax layer outputs a probability distribution is rather pleasing. In many problems it's convenient to be able to interpret the output activation $a^L_j$ as the network's estimate of the probability that the correct output is $j$. So, for instance, in the MNIST classification problem, we can interpret $a^L_j$ as the network's estimated probability that the correct digit classification is $j$.

By contrast, if the output layer was a sigmoid layer, then we certainly couldn't assume that the activations formed a probability distribution. I won't explicitly prove it, but it should be plausible that the activations from a sigmoid layer won't in general form a probability distribution. And so with a sigmoid output layer we don't have such a simple interpretation of the output activations.

\subsection*{Exercise}
\begin{itemize}
	\item Construct an example showing explicitly that in a network with a sigmoid output layer, the output activations $a^L_j$ won't always sum to 1.
\end{itemize}
We're starting to build up some feel for the softmax function and the way softmax layers behave. Just to review where we're at: the exponentials in Equation (\ref{eq:78}) ensure that all the output activations are positive. And the sum in the denominator of Equation (\ref{eq:78}) ensures that the softmax outputs sum to 1. So that particular form no longer appears so mysterious: rather, it is a natural way to ensure that the output activations form a probability distribution. You can think of softmax as a way of rescaling the $z^L_j$, and then squishing them together to form a probability distribution.

\subsection*{Exercises}
\begin{itemize}
\item \textbf{Monotonicity of softmax} Show that $\partial a^L_j / \partial z^L_k $is positive if $j=k$ and negative if $j\ne k$. As a consequence, increasing $z^L_j$ is guaranteed to increase the corresponding output activation, $a^L_j$, and will decrease all the other output activations. We already saw this empirically with the sliders, but this is a rigorous proof.
\item \textbf{Non-locality of softmax} A nice thing about sigmoid layers is that the output $a^L_j$ is a function of the corresponding weighted input, $a^L_j=\sigma(z^L_j)$. Explain why this is not the case for a softmax layer: any particular output activation $a^L_j$ depends on all the weighted inputs.
\end{itemize}
\subsection*{Problem}
\begin{itemize}
	\item \textbf{Inverting the softmax layer} Suppose we have a neural network with a softmax output layer, and the activations $a^L_j$ are known. Show that the corresponding weighted inputs have the form $z^L_j=\ln a^L_j+C$, for some constant $C$ that is independent of $j$.
\end{itemize}
\textbf{The learning slowdown problem:} We've now built up considerable familiarity with softmax layers of neurons. But we haven't yet seen how a softmax layer lets us address the learning slowdown problem. To understand that, let's define the \textit{log-likelihood} cost function. We'll use x to denote a training input to the network, and y to denote the corresponding desired output. Then the log-likelihood cost associated to this training input is
%C≡−lnaLy.(80)
\begin{equation}
	C \equiv -\ln a^L_j
\label{eq:80}
\end{equation}
So, for instance, if we're training with MNIST images, and input an image of a 7, then the log-likelihood cost is $-\ln a^L_7$. To see that this makes intuitive sense, consider the case when the network is doing a good job, that is, it is confident the input is a 7. In that case it will estimate a value for the corresponding probability $a^L_7$ which is close to 1, and so the cost $-\ln a^L_7$ will be small. By contrast, when the network isn't doing such a good job, the probability $a^L_7$ will be smaller, and the cost $-\ln a^L_7$ will be larger. So the log-likelihood cost behaves as we'd expect a cost function to behave.

What about the learning slowdown problem? To analyze that, recall that the key to the learning slowdown is the behaviour of the quantities $\partial C/\partial w^L_{jk}$ and $\partial C/\partial b^L_j$. I won't go through the derivation explicitly -- I'll ask you to do in the problems, below -- but with a little algebra you can show that\footnote{Note that I'm abusing notation here, using $y$ in a slightly different way to last paragraph. In the last paragraph we used $y$ to denote the desired output from the network -- e.g., output a ``7'' if an image of a 7 was input. But in the equations which follow I'm using $y$ to denote the vector of output activations which corresponds to 7, that is, a vector which is all 0s, except for a 1 in the 7th location.}
%\partial C\partial b^L_j\partial C\partial w^L_{jk}==a^L_j−yjaL−1k(a^L_j−yj)(81)(82)
\begin{eqnarray}
\frac{\partial C}{\partial b^L_j} & = & a^L_j-y_j  \label{eq:81}\\
\frac{\partial C}{\partial w^L_{jk}} & = & a^{L-1}_k (a^L_j-y_j)\label{eq:82}
\end{eqnarray}
These equations are the same as the analogous expressions obtained in our earlier analysis of the cross-entropy. Compare, for example, Equation (\ref{eq:82}) to Equation (\ref{eq:67}). It's the same equation, albeit in the latter I've averaged over training instances. And, just as in the earlier analysis, these expressions ensure that we will not encounter a learning slowdown. In fact, it's useful to think of a softmax output layer with log-likelihood cost as being quite similar to a sigmoid output layer with cross-entropy cost.

Given this similarity, should you use a sigmoid output layer and cross-entropy, or a softmax output layer and log-likelihood? In fact, in many situations both approaches work well. Through the remainder of this chapter we'll use a sigmoid output layer, with the cross-entropy cost. Later, in Chapter 6, we'll sometimes use a softmax output layer, with log-likelihood cost. The reason for the switch is to make some of our later networks more similar to networks found in certain influential academic papers. As a more general point of principle, softmax plus log-likelihood is worth using whenever you want to interpret the output activations as probabilities. That's not always a concern, but can be useful with classification problems (like MNIST) involving disjoint classes.

\subsection*{Problems}
\begin{itemize}
\item Derive Equations (\ref{eq:81}) and (\ref{eq:82}).
\item \textbf{Where does the ``softmax'' name come from?} Suppose we change the softmax function so the output activations are given by
\begin{equation}
	a^L_j = \frac{e^{c z^L_j}}{\sum_k e^{c z^L_k}},\label{eq:83}
\end{equation}
where $c$ is a positive constant. Note that $c=1$ corresponds to the standard softmax function. But if we use a different value of $c$ we get a different function, which is nonetheless qualitatively rather similar to the softmax. In particular, show that the output activations form a probability distribution, just as for the usual softmax. Suppose we allow $c$ to become large, i.e., $c\to \infty$. What is the limiting value for the output activations $a^L_j$? After solving this problem it should be clear to you why we think of the $c=1$ function as a ``softened'' version of the maximum function. This is the origin of the term ``softmax''.
\item \textbf{Backpropagation with softmax and the log-likelihood cost} In the last chapter we derived the backpropagation algorithm for a network containing sigmoid layers. To apply the algorithm to a network with a softmax layer we need to figure out an expression for the error $\delta^L_j \equiv \partial C / \partial z^L_j$ in the final layer. Show that a suitable expression is:
\begin{equation}
\delta^L_j = a^L_j -y_j. \label{eq:84}
\end{equation}
Using this expression we can apply the backpropagation algorithm to a network using a softmax output layer and the log-likelihood cost.
\end{itemize}

\section{Overfitting and regularization}
The Nobel prize winning physicist Enrico Fermi was once asked his opinion of a mathematical model some colleagues had proposed as the solution to an important unsolved physics problem. The model gave excellent agreement with experiment, but Fermi was skeptical. He asked how many free parameters could be set in the model. ``Four'' was the answer. Fermi replied\footnote{The quote comes from a charming article by Freeman Dyson, who is one of the people who proposed the flawed model. A four-parameter elephant may be found here.} : ``I remember my friend Johnny von Neumann used to say, with four parameters I can fit an elephant, and with five I can make him wiggle his trunk.''.

The point, of course, is that models with a large number of free parameters can describe an amazingly wide range of phenomena. Even if such a model agrees well with the available data, that doesn't make it a good model. It may just mean there's enough freedom in the model that it can describe almost any data set of the given size, without capturing any genuine insights into the underlying phenomenon. When that happens the model will work well for the existing data, but will fail to generalize to new situations. The true test of a model is its ability to make predictions in situations it hasn't been exposed to before.

Fermi and von Neumann were suspicious of models with four parameters. Our 30 hidden neuron network for classifying MNIST digits has nearly 24,000 parameters! That's a lot of parameters. Our 100 hidden neuron network has nearly 80,000 parameters, and state-of-the-art deep neural nets sometimes contain millions or even billions of parameters. Should we trust the results?

Let's sharpen this problem up by constructing a situation where our network does a bad job generalizing to new situations. We'll use our 30 hidden neuron network, with its 23,860 parameters. But we won't train the network using all 50,000 MNIST training images. Instead, we'll use just the first 1,000 training images. Using that restricted set will make the problem with generalization much more evident. We'll train in a similar way to before, using the cross-entropy cost function, with a learning rate of $\eta=0.5$ and a mini-batch size of 10. However, we'll train for 400 epochs, a somewhat larger number than before, because we're not using as many training examples. Let's use \texttt{network2} to look at the way the cost function changes:


\begin{lstlisting}

>>> import mnist_loader 
>>> training_data, validation_data, test_data = mnist_loader.load_data_wrapper()
>>> import network2 
>>> net = network2.Network([784, 30, 10], cost=network2.CrossEntropyCost) 
>>> net.large_weight_initializer()
>>> net.SGD(training_data[:1000], 400, 10, 0.5, evaluation_data=test_data, monitor_evaluation_accuracy=True, monitor_training_cost=True)
\end{lstlisting}
Using the results we can plot the way the cost changes as the network learns\footnote{This and the next four graphs were generated by the program \texttt{overfitting.py}.} :

\begin{center}
	\includegraphics[width=0.7\linewidth]{figures/ch3/overfitting1}
\end{center}
This looks encouraging, showing a smooth decrease in the cost, just as we expect. Note that I've only shown training epochs 200 through 399. This gives us a nice up-close view of the later stages of learning, which, as we'll see, turns out to be where the interesting action is.

Let's now look at how the classification accuracy on the test data changes over time:
\begin{center}
	\includegraphics[width=0.7\linewidth]{figures/ch3/overfitting2}
\end{center}
Again, I've zoomed in quite a bit. In the first 200 epochs (not shown) the accuracy rises to just under 82 percent. The learning then gradually slows down. Finally, at around epoch 280 the classification accuracy pretty much stops improving. Later epochs merely see small stochastic fluctuations near the value of the accuracy at epoch 280. Contrast this with the earlier graph, where the cost associated to the training data continues to smoothly drop. If we just look at that cost, it appears that our model is still getting ``better''. But the test accuracy results show the improvement is an illusion. Just like the model that Fermi disliked, what our network learns after epoch 280 no longer generalizes to the test data. And so it's not useful learning. We say the network is \textit{overfitting} or \textit{overtraining} beyond epoch 280.

You might wonder if the problem here is that I'm looking at the cost on the training data, as opposed to the \textit{classification accuracy} on the test data. In other words, maybe the problem is that we're making an apples and oranges comparison. What would happen if we compared the cost on the training data with the cost on the test data, so we're comparing similar measures? Or perhaps we could compare the classification accuracy on both the training data and the test data? In fact, essentially the same phenomenon shows up no matter how we do the comparison. The details do change, however. For instance, let's look at the cost on the test data:
\begin{center}
	\includegraphics[width=0.7\linewidth]{figures/ch3/overfitting3}
\end{center}
We can see that the cost on the test data improves until around epoch 15, but after that it actually starts to get worse, even though the cost on the training data is continuing to get better. This is another sign that our model is overfitting. It poses a puzzle, though, which is whether we should regard epoch 15 or epoch 280 as the point at which overfitting is coming to dominate learning? From a practical point of view, what we really care about is improving classification accuracy on the test data, while the cost on the test data is no more than a proxy for classification accuracy. And so it makes most sense to regard epoch 280 as the point beyond which overfitting is dominating learning in our neural network.

Another sign of overfitting may be seen in the classification accuracy on the training data:
\begin{center}
	\includegraphics[width=0.7\linewidth]{figures/ch3/overfitting4}
\end{center}
The accuracy rises all the way up to 100 percent. That is, our network correctly classifies all 1,000 training images! Meanwhile, our test accuracy tops out at just 82.27 percent. So our network really is learning about peculiarities of the training set, not just recognizing digits in general. It's almost as though our network is merely memorizing the training set, without understanding digits well enough to generalize to the test set.

Overfitting is a major problem in neural networks. This is especially true in modern networks, which often have very large numbers of weights and biases. To train effectively, we need a way of detecting when overfitting is going on, so we don't overtrain. And we'd like to have techniques for reducing the effects of overfitting.

The obvious way to detect overfitting is to use the approach above, keeping track of accuracy on the test data as our network trains. If we see that the accuracy on the test data is no longer improving, then we should stop training. Of course, strictly speaking, this is not necessarily a sign of overfitting. It might be that accuracy on the test data and the training data both stop improving at the same time. Still, adopting this strategy will prevent overfitting.

In fact, we'll use a variation on this strategy. Recall that when we load in the MNIST data we load in three data sets:
\begin{lstlisting}
>>> import mnist_loader 
>>> training_data, validation_data, test_data = mnist_loader.load_data_wrapper()

\end{lstlisting}
Up to now we've been using the \texttt{training\_data} and \texttt{test\_data}, and ignoring the \texttt{validation\_data}. The \texttt{validation\_data} contains 10,000 images of digits, images which are different from the 50,000 images in the MNIST training set, and the 10,000 images in the MNIST test set. Instead of using the \texttt{test\_data} to prevent overfitting, we will use the \texttt{validation\_data}. To do this, we'll use much the same strategy as was described above for the test\_data. That is, we'll compute the classification accuracy on the \texttt{validation\_data} at the end of each epoch. Once the classification accuracy on the \texttt{validation\_data} has saturated, we stop training. This strategy is called early stopping. Of course, in practice we won't immediately know when the accuracy has saturated. Instead, we continue training until we're confident that the accuracy has saturated\footnote{It requires some judgment to determine when to stop. In my earlier graphs I identified epoch 280 as the place at which accuracy saturated. It's possible that was too pessimistic. Neural networks sometimes plateau for a while in training, before continuing to improve. I wouldn't be surprised if more learning could have occurred even after epoch 400, although the magnitude of any further improvement would likely be small. So it's possible to adopt more or less aggressive strategies for early stopping.}.

Why use the \texttt{validation\_data} to prevent overfitting, rather than the \texttt{test\_data}? In fact, this is part of a more general strategy, which is to use the \texttt{validation\_data} to evaluate different trial choices of hyper-parameters such as the number of epochs to train for, the learning rate, the best network architecture, and so on. We use such evaluations to find and set good values for the hyper-parameters. Indeed, although I haven't mentioned it until now, that is, in part, how I arrived at the hyper-parameter choices made earlier in this book. (More on this later.)

Of course, that doesn't in any way answer the question of why we're using the \texttt{validation\_data} to prevent overfitting, rather than the \texttt{test\_data}. Instead, it replaces it with a more general question, which is why we're using the \texttt{validation\_data} rather than the \texttt{test\_data} to set good hyper-parameters? To understand why, consider that when setting hyper-parameters we're likely to try many different choices for the hyper-parameters. If we set the hyper-parameters based on evaluations of the test\_data it's possible we'll end up overfitting our hyper-parameters to the \texttt{test\_data}. That is, we may end up finding hyper-parameters which fit particular peculiarities of the \texttt{test\_data}, but where the performance of the network won't generalize to other data sets. We guard against that by figuring out the hyper-parameters using the \texttt{validation\_data}. Then, once we've got the hyper-parameters we want, we do a final evaluation of accuracy using the \texttt{test\_data}. That gives us confidence that our results on the \texttt{test\_data} are a true measure of how well our neural network generalizes. To put it another way, you can think of the validation data as a type of training data that helps us learn good hyper-parameters. This approach to finding good hyper-parameters is sometimes known as the hold out method, since the \texttt{validation\_data} is kept apart or ``held out'' from the \texttt{training\_data}.

Now, in practice, even after evaluating performance on the \texttt{test\_data} we may change our minds and want to try another approach - perhaps a different network architecture - which will involve finding a new set of hyper-parameters. If we do this, isn't there a danger we'll end up overfitting to the \texttt{test\_data} as well? Do we need a potentially infinite regress of data sets, so we can be confident our results will generalize? Addressing this concern fully is a deep and difficult problem. But for our practical purposes, we're not going to worry too much about this question. Instead, we'll plunge ahead, using the basic hold out method, based on the \texttt{training\_data}, \texttt{validation\_data}, and \texttt{test\_data}, as described above.

We've been looking so far at overfitting when we're just using 1,000 training images. What happens when we use the full training set of 50,000 images? We'll keep all the other parameters the same (30 hidden neurons, learning rate 0.5, mini-batch size of 10), but train using all 50,000 images for 30 epochs. Here's a graph showing the results for the classification accuracy on both the training data and the test data. Note that I've used the test data here, rather than the validation data, in order to make the results more directly comparable with the earlier graphs.
\begin{center}
	\includegraphics[width=0.7\linewidth]{figures/ch3/overfitting_full}
\end{center}
As you can see, the accuracy on the test and training data remain much closer together than when we were using 1,000 training examples. In particular, the best classification accuracy of 97.86 percent on the training data is only 2.53 percent higher than the 95.33 percent on the test data. That's compared to the 17.73 percent gap we had earlier! Overfitting is still going on, but it's been greatly reduced. Our network is generalizing much better from the training data to the test data. In general, one of the best ways of reducing overfitting is to increase the size of the training data. With enough training data it is difficult for even a very large network to overfit. Unfortunately, training data can be expensive or difficult to acquire, so this is not always a practical option.

\section{Regularization}
Increasing the amount of training data is one way of reducing overfitting. Are there other ways we can reduce the extent to which overfitting occurs? One possible approach is to reduce the size of our network. However, large networks have the potential to be more powerful than small networks, and so this is an option we'd only adopt reluctantly.

Fortunately, there are other techniques which can reduce overfitting, even when we have a fixed network and fixed training data. These are known as \textit{regularization} techniques. In this section I describe one of the most commonly used regularization techniques, a technique sometimes known as \textit{weight decay} or \textit{L2 regularization}. The idea of L2 regularization is to add an extra term to the cost function, a term called the \textit{regularization term}. Here's the regularized cross-entropy:
\begin{eqnarray}
C = -\frac{1}{n} \sum_{xj} \left[ y_j \ln a^L_j+(1-y_j) \ln (1-a^L_j)\right] + \frac{\lambda}{2n} \sum_w w^2.
\label{eq:85}
\end{eqnarray}
%C=−1n∑xj[yjlnaLj+(1−yj)ln(1−aLj)]+\lambda 2n∑ww2.(85)
The first term is just the usual expression for the cross-entropy. But we've added a second term, namely the sum of the squares of all the weights in the network. This is scaled by a factor $\lambda/2n$, where $\lambda>0$ is known as the \textit{regularization parameter}, and $n$ is, as usual, the size of our training set. I'll discuss later how $\lambda$ is chosen. It's also worth noting that the regularization term doesn't include the biases. I'll also come back to that below.

Of course, it's possible to regularize other cost functions, such as the quadratic cost. This can be done in a similar way:
\begin{eqnarray} C = \frac{1}{2n} \sum_x \|y-a^L\|^2 + \frac{\lambda}{2n} \sum_w w^2.
\label{eq:86}\end{eqnarray}
%C=12n∑x∥y−aL∥2+\lambda 2n∑ww2.(86)
In both cases we can write the regularized cost function as
%C=C0+\lambda 2n∑ww2,(87)
\begin{eqnarray}  C = C_0 + \frac{\lambda}{2n}\sum_w w^2,
\label{eq:87}
\end{eqnarray}
where $C_0$ is the original, unregularized cost function.

Intuitively, the effect of regularization is to make it so the network prefers to learn small weights, all other things being equal. Large weights will only be allowed if they considerably improve the first part of the cost function. Put another way, regularization can be viewed as a way of compromising between finding small weights and minimizing the original cost function. The relative importance of the two elements of the compromise depends on the value of $\lambda$: when $\lambda$  is small we prefer to minimize the original cost function, but when $\lambda$ is large we prefer small weights.

Now, it's really not at all obvious why making this kind of compromise should help reduce overfitting! But it turns out that it does. We'll address the question of why it helps in the next section. But first, let's work through an example showing that regularization really does reduce overfitting.

To construct such an example, we first need to figure out how to apply our stochastic gradient descent learning algorithm in a regularized neural network. In particular, we need to know how to compute the partial derivatives $\partial{}C/\partial{}w$ and $\partial{}C/\partial{}b$ for all the weights and biases in the network. Taking the partial derivatives of Equation (\ref{eq:87}) gives
\begin{eqnarray} 
\frac{\partial C}{\partial w} & = & \frac{\partial C_0}{\partial w} + \frac{\lambda}{n} w \label{eq:88}\\ 
\frac{\partial C}{\partial b} & = & \frac{\partial C_0}{\partial b}. \label{eq:89}\end{eqnarray}
%\partial{}C\partial{}w\partial{}C\partial{}b==\partial{}C0\partial{}w+\lambda nw\partial{}C0\partial{}b.(88)(89)
The $\partial{}C_0/\partial{}w$ and $\partial{}C_0/\partial{}b$ terms can be computed using backpropagation, as described in the last chapter. And so we see that it's easy to compute the gradient of the regularized cost function: just use backpropagation, as usual, and then add $\frac{\lambda}{n}w$ to the partial derivative of all the weight terms. The partial derivatives with respect to the biases are unchanged, and so the gradient descent learning rule for the biases doesn't change from the usual rule:
\begin{equation}
b\to b-\eta\frac{\partial{}C_0}{\partial{}b}.
\label{eq:90}
\end{equation}
The learning rule for the weights becomes:
\begin{eqnarray} 
w & \rightarrow & w-\eta \frac{\partial C_0}{\partial w}-\frac{\eta \lambda}{n} w  = \left(1-\frac{\eta \lambda}{n}\right) w -\eta \frac{\partial	C_0}{\partial w}. 
\label{eq:92}
\end{eqnarray}
%w→=w−\eta\partial{}C0\partial{}w−\eta\lambda nw(1−\eta\lambda n)w−\eta\partial{}C0\partial{}w.(91)(92)
This is exactly the same as the usual gradient descent learning rule, except we first rescale the weight $w$ by a factor $1−\eta\frac{\lambda}{n}$. This rescaling is sometimes referred to as \textit{weight decay}, since it makes the weights smaller. At first glance it looks as though this means the weights are being driven unstoppably toward zero. But that's not right, since the other term may lead the weights to increase, if so doing causes a decrease in the unregularized cost function.

Okay, that's how gradient descent works. What about stochastic gradient descent? Well, just as in unregularized stochastic gradient descent, we can estimate $\partial{}C_0/\partial{}w$ by averaging over a mini-batch of $m$ training examples. Thus the regularized learning rule for stochastic gradient descent becomes (c.f. Equation (\ref{eq:20}))
\begin{eqnarray} 
w \rightarrow \left(1-\frac{\eta \lambda}{n}\right) w -\frac{\eta}{m} \sum_x \frac{\partial C_x}{\partial w}, 
\label{eq:93}
\end{eqnarray}
%w→(1−\eta\lambda n)w−\etam∑x\partial{}Cx\partial{}w,(93)
where the sum is over training examples $x$ in the mini-batch, and $C_x$ is the (unregularized) cost for each training example. This is exactly the same as the usual rule for stochastic gradient descent, except for the $1−\eta\lambda/n$ weight decay factor. Finally, and for completeness, let me state the regularized learning rule for the biases. This is, of course, exactly the same as in the unregularized case (c.f. Equation (\ref{eq:21})),
\begin{eqnarray}
	b \to b - \frac{\eta}{m} \sum_x \frac{\partial C_x}{\partial b},
	\label{eq:94}
\end{eqnarray}
%b→b−\etam∑x\partial{}Cx\partial{}b,(94)
where the sum is over training examples $x$ in the mini-batch.

Let's see how regularization changes the performance of our neural network. We'll use a network with 30 hidden neurons, a mini-batch size of 10, a learning rate of 0.5, and the cross-entropy cost function. However, this time we'll use a regularization parameter of $\lambda =0.1$. Note that in the code, we use the variable name \texttt{lmbda}, because \texttt{lambda} is a reserved word in Python, with an unrelated meaning. I've also used the \texttt{test\_data} again, not the \texttt{validation\_data}. Strictly speaking, we should use the \texttt{validation\_data}, for all the reasons we discussed earlier. But I decided to use the \texttt{test\_data} because it makes the results more directly comparable with our earlier, unregularized results. You can easily change the code to use the \texttt{validation\_data} instead, and you'll find that it gives similar results.

\begin{lstlisting}

>>> import mnist_loader 
>>> training_data, validation_data, test_data =  mnist_loader.load_data_wrapper() 
>>> import network2 
>>> net = network2.Network([784, 30, 10], cost=network2.CrossEntropyCost)
>>> net.large_weight_initializer()
>>> net.SGD(training_data[:1000], 400, 10, 0.5, evaluation_data=test_data, lmbda = 0.1, \
... monitor_evaluation_cost=True, monitor_evaluation_accuracy=True, monitor_training_cost=True, monitor_training_accuracy=True)

\end{lstlisting}
The cost on the training data decreases over the whole time, much as it did in the earlier, unregularized case\footnote{This and the next two graphs were produced with the program \texttt{overfitting.py.}}:
\begin{center}
	\includegraphics[width=0.7\linewidth]{figures/ch3/regularized1}
\end{center}
But this time the accuracy on the \texttt{test\_data} continues to increase for the entire 400 epochs:
\begin{center}
	\includegraphics[width=0.7\linewidth]{figures/ch3/regularized2}
\end{center}
Clearly, the use of regularization has suppressed overfitting. What's more, the accuracy is considerably higher, with a peak classification accuracy of 87.1 percent, compared to the peak of 82.27 percent obtained in the unregularized case. Indeed, we could almost certainly get considerably better results by continuing to train past 400 epochs. It seems that, empirically, regularization is causing our network to generalize better, and considerably reducing the effects of overfitting.

What happens if we move out of the artificial environment of just having 1,000 training images, and return to the full 50,000 image training set? Of course, we've seen already that overfitting is much less of a problem with the full 50,000 images. Does regularization help any further? Let's keep the hyper-parameters the same as before -- 30 epochs, learning rate 0.5, mini-batch size of 10. However, we need to modify the regularization parameter. The reason is because the size n of the training set has changed from $n$=1,000 to $n$=50,000, and this changes the weight decay factor $1−\eta\lambda/n$. If we continued to use $\lambda =0.1$ that would mean much less weight decay, and thus much less of a regularization effect. We compensate by changing to $\lambda =5.0$.

Okay, let's train our network, stopping first to re-initialize the weights:

\begin{lstlisting}

>>> net.large_weight_initializer()
>>> net.SGD(training_data, 30, 10, 0.5, evaluation_data=test_data, lmbda = 5.0,
... monitor_evaluation_accuracy=True, monitor_training_accuracy=True)

\end{lstlisting}
We obtain the results:
\begin{center}
	\includegraphics[width=0.7\linewidth]{figures/ch3/regularized_full}
\end{center}
There's lots of good news here. First, our classification accuracy on the test data is up, from 95.49 percent when running unregularized, to 96.49 percent. That's a big improvement. Second, we can see that the gap between results on the training and test data is much narrower than before, running at under a percent. That's still a significant gap, but we've obviously made substantial progress reducing overfitting.


Finally, let's see what test classification accuracy we get when we use 100 hidden neurons and a regularization parameter of $\lambda =5.0$. I won't go through a detailed analysis of overfitting here, this is purely for fun, just to see how high an accuracy we can get when we use our new tricks: the cross-entropy cost function and L2 regularization.

\begin{lstlisting}

>>> net = network2.Network([784, 100, 10], cost=network2.CrossEntropyCost)
>>> net.large_weight_initializer()
>>> net.SGD(training_data, 30, 10, 0.5, lmbda=5.0, evaluation_data=validation_data,
... monitor_evaluation_accuracy=True)

\end{lstlisting}
The final result is a classification accuracy of 97.92 percent on the validation data. That's a big jump from the 30 hidden neuron case. In fact, tuning just a little more, to run for 60 epochs at $\eta=0.1$ and $\lambda =5.0$ we break the 98 percent barrier, achieving 98.04 percent classification accuracy on the validation data. Not bad for what turns out to be 152 lines of code!

I've described regularization as a way to reduce overfitting and to increase classification accuracies. In fact, that's not the only benefit. Empirically, when doing multiple runs of our MNIST networks, but with different (random) weight initializations, I've found that the unregularized runs will occasionally get ``stuck'', apparently caught in local minima of the cost function. The result is that different runs sometimes provide quite different results. By contrast, the regularized runs have provided much more easily replicable results.

Why is this going on? Heuristically, if the cost function is unregularized, then the length of the weight vector is likely to grow, all other things being equal. Over time this can lead to the weight vector being very large indeed. This can cause the weight vector to get stuck pointing in more or less the same direction, since changes due to gradient descent only make tiny changes to the direction, when the length is long. I believe this phenomenon is making it hard for our learning algorithm to properly explore the weight space, and consequently harder to find good minima of the cost function.
\end{document}